\documentclass{amsart}
\usepackage{amsmath, amsthm, amssymb}
%
\title{Review: Geoffrey Hellman, \emph{Mathematics and its Logics}}
%
\author{Chris Scambler 
\\
All Souls College\\
University of Oxford}
%
% These will be typeset in italics
\newtheorem{Theorem}{Theorem}[section]
\newtheorem{Proposition}[Theorem]{Proposition}
\newtheorem{Lemma}[Theorem]{Lemma}
\newtheorem{Corollary}[Theorem]{Corollary}

% These will be typeset in Roman
\theoremstyle{definition}
\newtheorem{Definition}[Theorem]{Definition}
\newtheorem{Fact}[Theorem]{Fact}
\newtheorem{Conjecture}[Theorem]{Conjecture}
\newtheorem{Remark}[Theorem]{Remark}
%
\begin{document} 
\maketitle
%\begin{abstract} \end{abstract}

\section{Introduction}

\emph{Mathematics and its logics} is a collection of Geoffrey Hellman's papers. 
The volume covers a huge variety of topics:
there are essays on the foundational role of category theory, 
on the nature and value of restrictive programs in foundations of mathematics such as nominalism and predicativism, 
on modal structuralism, on the Quine-Carnap debate, 
on intuitionism and smooth infinitesimal analysis, 
and much more besides. 
Hellman's important contributions on all of these topics are reflected in the collected papers, 
and in addition they tend to contain excellent introductions to the relevant debates along the way.
As a result, 
the book would make a valuable addition to the library of any graduate student or professional with a specialization in philosophy of mathematics. 

The book is split into three parts reflecting different aspects of Hellman's work. 
Each part consists of a series of papers with a common theme.
Most of the papers are very well known, have been highly influential, 
and have already been subjects of much discussion in the literature.
However, in addition, 
each part also offers a newly written paper.
One of these is a critical piece, responding to recent work by Penelope Maddy. 
The other two push at the frontiers of the existing work showcased in parts one and two.
Since the older works are already so well known and have been so extensively discussed, 
after summarizing the papers offered in each part, 
in this review I will go on to discuss the contrubtions made by the two new non-critical papers.

\section{Overview}

Part I consists in papers related to Hellman's modal structuralism and surrounding issues.
The first paper of part I, \emph{Structuralism without Structures}, 
is one of the founding works in the theory of modal structuralism;
in it, mathematical foundations are developed in detail for many areas of mathematics
in modal structuralist terms, and the philosophical credentials of modal structuralism are assessed.
(The paper also contains interesting points of comparison between modal structuralism and the predicativst viewpoint, 
which is itself the topic of scrutiny in part II.) 
The second paper, 
\emph{What Is Categorical Structuralism?}, 
presents a comprehensive discussion of the role of category theory in foundations of mathematics, 
arguing 
(against Colin McClarty and Steve Awodey) 
that ultimately category theory is in an important sense lacking in this regard. 
Papers three and four, 
\emph{On the Significance of the Burali Forti Paradox} 
and \emph{Extending the Iterative Concept of Set}, 
discuss the set-theoretic paradoxes from the potentialist point of view, 
making the case that the `height potentialist' perspective implied by modal structuralism allows for a more satisfying solution 
to the set-theoretic paradoxes,
 as well as clearer justifications for certain standard axioms of set theory, 
than are afforded by the more common `actualist' view. 
Papers five and six are related to nominalism, 
and (among other things) are concerned to defend the particular form of nominalism implied by his modal structuralism.

Part II offers a series of four papers 
exploring the nature and value of predicative foundations for mathematics. 
The first paper, \emph{Predicative Foundations of Arithmetic}, 
is joint with Solomon Feferman. 
It develops the first-order theory of arithmetic and proves its categoricity 
using an axiom system they call $\mathsf{EFSC}$, 
the `elementary theory of finite sets and classes',
which is argued also to be predicatively acceptable.
The framework developed presents a beatuifully economic account 
of the natural number structure from a remarkably austere collection of principles. 
In addition, the presented results may also be of relevance to contemporary discussions in philosophy.
The recent discussion of `internal categoricity' results and their philosophical significance, 
for instance, 
tends to take it as read that \emph{impredicative} comprehension is needed
to secure categoricity in this way, 
whereas the work done by Feferman and Hellman
seems to suggest that one can do with less.
A little bit -- 
in the form of $\mathsf{EFSC}$ -- 
goes further than one might expect. 

The second paper of part II, 
\emph{Challenges to Predicative Foundations of Arithmetic},
is also joint with Feferman.
It considers various philosophical objections raised to predicativism in the light 
of the mathematical work done in the first paper, 
and also presents some further technical developments contributed by Peter Aczel. 
The third paper, \emph{Predicativsm as a Philosophical Position},
is sole-authored. 
Perhaps unsurprisingly, 
it presents a more critical examination of predicativism as a philosophical position,
arguing that ultimately the central contribution of predicative foundations
is to emphasize the 
\emph{dispensability} of the uncountable from the point of view of modern science,
rather than providing an independently coherent and plausible foundation for mathematics.
The fourth paper, \emph{On the G\"odel-Friedman program}, 
is newly written for the volume.
It continues the analysis of the previous article in a more general context,
in particular that of Harvey Friedman's program for justifying large cardinal axioms 
by appeal to their utility in solving undecidable problems at the level of the naturals. 
Beyond the insightful discussion of Friedman's program, 
the article moves into novel and interesting territory regarding
the epistomology of large cardinals,
and in particular sets out the outlines of a Bayesian approach to justification here.

Part III is a little less homogeneous than the previous two parts. 
It presents five papers in the domain of philosophical logic. The first, 
\emph{Logical Truth by Linguistic Convention}, 
offers a defense of the Carnapian thesis 
that first-order logical knowledge is in a significant sense `analytic'. 
The second, \emph{Never say ``Never''!}, 
argues that intuitionism is an important sense not self-contained, 
since it turns out to rely on a `non-intuitiontistic' understanding of infintary quantification.
The third paper, 
\emph{Constructive Mathematics and Quantum Mechanics: Unbounded Operators and the Spectral Theorem} 
continues the assault on intuitionism by arguing that constructive mathematics 
is incapable of proving results needed in the foundations of quantum mechanics, 
especially the so-called `spectral theorem'. 
Paper four, \emph{If ``If-Then'' then what?}, discusses Maddy's recent attempts to revive (and extend) 
the classical position of `If-Then-ism' in foundations of mathematics.
Finally, the fifth paper, 
\emph{Mathematical Pluralism: The Case of Smooth Infinitesimal Analysis}, 
discusses the issue of how to understand the logical connectives in SIA, 
and argues that ultimately they must be understood as only as implicitly defined 
relative to a presupposed classical meta-theory, 
thus amplifying the theme in this part that intutionism and intuitionistic theories are, 
taken independently of a classical meta-translation, 
ultimately inadequate.

As this summary makes clear, the book addresses a huge 
(and mouthwatering) 
selection of topics, too many to sensibly discuss here. Instead, 
I will now turn to a discussion of two points of interest 
arising in two of the newly written papers published in the volume.

\section{On the justification of infinity}

One of the central claims of the fourth paper of part I, 
\emph{Extending the Iterative Concept of Set}, 
is that the `height potentialist' in set theory 
is able to offer a better justification for 
the axioms of infinity and replacement
than is avaiable on the standard `height actualist' framework.

The stage is set with Boolos' famous paper,
\emph{The iterative concept of set},
as the backdrop. There, 
Boolos shows how certain axioms of standard set theory 
can be motivated by appeal to 
the idea that sets are to be thought of as 
`formed in a well-ordered series of stages',
wherein at each `stage' of the process of set-formation 
one forms all possible sets 
of things formed at prior stages. 

However,
not all axioms of standard set theory were derivable 
from Boolos' axioms. Saliently, Boolos abandons the axiom of replacement 
altogether; more subtly, in order to secure infinity, 
he has to write in the existence of an infinite stage as a new axiom 
in the stage theory.

Like all standard axiomatic set theories, Boolos' stage theory is
\emph{(height) actualist}. This means that it doesn't make room for 
a substantive notion of a \emph{merely possible} set or stage of set formation.
Rather, there simply are the sets and stages of formation that there are;
there is no sensible question to ask about whether or not there might in addition be
more possible sets or stages.

The \emph{(height) potentialist} framework, on the other hand, 
says that no matter what sets there are there could always be ones of higher rank;
in terms of stages, this says that 
no matter what stages of construction have been carried out,
more such stages are possible. The idea is one of a never ending, 
`potential hierarchy' of sets, as opposed to the standard actualist 
idea of The Cumulative Hierarchy.

The paper at issue makes the striking suggestion that height potentialism
might be able to do better 
when it comes to justifying infinity and replacement 
than Boolos or the height actualist more generally is able to. 
Now as a matter of fact I agree with Hellman's claim here;
but I am not convinced by his argument.
Let me explain why. For the sake of simplicity, I will focus my discussion on infinity; 
analogous though more tedious-to-state points stand to be made 
about replacement as well.

How might height potentialism help with deriving the axiom of infinity? 
First, it is important to note that in one very mundane (but important) sense,
it necessarily cannot help. For the standard axiom of infinity says, 
simply and unapologetically, 
that there \emph{exists} an infinite set. And many height potentialist theories, 
Hellman's own brands included, 
tend not to have this as a consequence, and inded they tend not 
to even imply there exist \emph{any sets at all}. 
On potentialist theories, in general, sets are possibilia, 
and so the relevant `axiom of infinity' does not assert that 
infinite sets \emph{exist}, but rather 
only the `potentialist translation' of the standard axiom, 
that infinite sets \emph{are possible}. 
This latter is really the principle 
whose justification is at issue in the paper.\footnote{ 
    Hellman does not this in the article, but in a rather off hand way;
    I think perhaps it might have been emphaszied more, and earlier.
    }

Now is probably a good time to explain why 
\emph{I} 
believe that the height potentialist 
has available a justification for the axiom of infinity 
not available to the actualist. 
That is simply because their axiom is a 
\emph{ weakening } 
of the corresponding actualist principle. 
The height potentialist only needs to secure the 
\emph{possible} 
truth 
(in whatever sense is at issue) 
of the actualist's axiom. So obviously routes of justification 
will be open to one but not the other, though exactly how this plays out 
will depend on the brand of potentialism (and the sense of possibility) at issue.

But this simple thought is not Hellman's. 
According to Hellman, 
there are principles statable in the language of his modal stage theory 
-- essentially, that of modal, plural logic with 
vocabulary for stages and sets --
that are at once significantly more self-evident
than the potentialist's axiom of infinity 
and that nevertheless imply it as a consequence. 
It is this claim I am somewhat doubtful of.

Hellman considers one axiom of this kind 
that he ultimately rejects on grounds of its being `too close' 
to the sought conclusion to really gain any ground. This is the assertion:
\begin{description}
\item[A] It is possible that there are some stages $ss$ 
such that the empty set is formed at one of them, 
and every stage in the $ss$ is succeeded by another stage in the $ss$.\footnote{
    Following Hellman, I will use double letters $xx$ to range over pluralities, 
    and will tend to use $s$, $ss$ when the intended range is stages.
    }
\end{description}
In the context of the rest of the stage theory, 
and in particular the axiom that every stage contains all possible sets 
from among things at the previous stage,
this readily implies there is a least stage with infinitely many stages before it, 
and with it many infinite sets. But, according to Hellman, this axiom is 
\begin{quote} ``Too close for comfort to the conclusion sought. 
    Once you have an infinity of compossible things... 
    not surprisingly you generate a set of those things at the next stage.''\end{quote}
What then is Hellman's alternative? To introduce it, 
we need the following definition. 
Say that a property $P(ss)$ that applies to stages $ss$ is
 \emph{Indefinitely Extendible} (IE) 
iff each of

\begin{description} 
    \item[B]    $P(ss)$ is possible,
    \item[C]    given any possible stages $ss$ with $P(ss)$, 
                it is possible to find a $t$ after all $ss$ with $P(ss + t)$
\end{description}
are true. In the latter of course $ss + t$ is just 
the $ss$ together with $t$ (which will be supplied by 
our plural comprehension axioms). The \emph{rough} idea is that $P$ is 
an IE property when it is always possible to find further 
stages to which it applies: so no matter how far you go on 
in the process of set formation, you will always eventually 
find new series of stages that satisfy $P$.

Now, let $N$ be the predicate
\begin{quote} Every stage in the $ss$ is one at which a natural number is formed
\end{quote} 
Then the axiom Hellman proposes is  
\begin{description}
    \item[HA] $N$ is not IE
\end{description}  
Let us see what {\bf HA} amounts to. Substituting $N$ for $P$ in $IE$ 
and negating we get the negation of one of \emph{B} or \emph{C} with $N = P$. 
The resulting negation of \emph{B} is demonstrably false 
given the rest of Hellman's stage theory, 
so we are left with (up to trivial grammatical reshuffling)

\begin{description}
    \item[HA$^+$] Possibly, there are stages $ss$ 
    where every one of $ss$ witnesses the formation of a new natural number, 
    and such that no later stage witnesses the formation of a natural number.
\end{description}

Using the fact that the rest of Hellman's theory implies 
it is possible for there to be a stage $s_n$ at which 
each particular natural number $n$ is formed, 
one can easily go on to show that there must possibly be a stage 
at which the set of all natural numbers is formed (and much more besides). 
But this is the potentialist version of the axiom of infinity we set out to secure.

I have two issues with this argument. 
The first (and less significant) issue is that the machinery of stage theory 
and the Boolosian iterative conception seems superfluous, so that it is 
rather unobvious why talk of stages and the extra vocabulary this requires should 
be relevant to the argument. (In fact, shorn of commitment to stages, Hellman's 
axioms closely resemble the axioms of infinity and replacement as they are 
presented in Linnebo's potentialist set theory.) 
The second and more significant point is that it is seems there is in fact 
little daylight between {\bf HA} (or {\bf HA$^+$})
and axiom {\bf A}, so that if the one is  `too close for comfort' to the 
sought conclusion then so is the other.

As to the first point, it is a little unclear why talk of stages 
should be relevant to such arguments for the justification of infinity, 
and in fact there are simpler though intimately related proposals known 
in the literature. For example, a simpler definition of an IE predicate says 
that the predicate $P$ is IE if necessarily, given any plurality $pp$ of $P$-ers, 
it is possible to find a $p$ not among the $pp$ with $P(p)$. 
Then the axiom that says the predicate `is a natural number' or 
`is a hereditarily finite set' is not IE in this sense seems to have 
all the plausibility of Hellman's proposal but without need to refer to stages. 
Indeed, as I just mentioned, this is essentially the statement of the axiom
of infinity given in Linnebo (cite). But now why should it be thought that the postulation
that the stages of natural number formation are not indefinitely extendible is any 
more plausible than the simpler claim that that the natural numbers or the hereditarily 
finite sets are not indefinitely extendible in this sense?

This first issue is really not a big deal; either way of putting the argument
is fine, it just seems a little strange to use the more complicated 
vocabulary when not strictly necessary. But the more pressing issues
regards the extent to which {\bf A} and {\bf HA} differ significantly 
enough for cognitive ground to be gained. And here I am somewhat skeptical.

Let's review the dialectic. Hellman's stage theory is committed to 

\begin{description} 
    \item[E] For each natural number $n$, 
    it is possible to find a stage $s_n$ at which the 
    finite Von Neumann ordinal $n$ is formed.\footnote{
        This is a slightly loose and tendentious way to put things,
        since it appeals to natural numbers in the metatheory. But
        the same points can be made, in a more subtle way, without
        this crutch.
    }
\end{description} 
The question is whether in addition it is possible to form 
all such stages `at once'. {\bf A} says `yes' quite straightforwardly: 
it asserts that it is possible to run through all the finite stages 
of the set construction process. In the context of {\bf E} on the other hand, 
{\bf D} is a way of saying `yes' bentbackwardly, by saying that eventually 
the stages that witness formation of natural numbers \emph{stop coming}, 
even though (as the stage theory requires) the stages continue. 
But (given {\bf E}) this really seems a trivial restatement of the 
claim that eventually you get all of them.  One can say eventually you get 
all possible stages at which natural numbers 
are formed; or, that eventually you necessarily won't get any more stages at which natural 
numbers are formed. But in either case,
so long as you also have that every natural 
number is formed at some possible stage,
you are saying essentially the same thing, and it is therefore rather 
hard to imagine anyone uncertain of the one statement being 
reassured by appeal to the other.

In my view then axioms {\bf A} and {\bf HA} come to more or less the 
same thing, so if one is `too close' to the sought conclusion to really advance 
the debate then so is the other. In fact I think it highly unlikely that an axiom 
of more intrinsic plausibility than the simple assertion that an infinite set is possible 
will be found that nevertheless implies it in the potentialist's logical framework.  
This is not to say, though, that the potentialist is no better off than the actualist, 
since as I have already stressed they will in general have a simpler task in justifying 
their weaker form of the axiom of infinity. If all you need is the \emph{logical possibility}
(say) of an infinity of sets existing together, you'll be better off than if you 
need to argue for the \emph{actual existence} of an infinity of sets.

\section{On Bayesianism and Large Cardinals}

The second theme I wanted to pick up on in this review concerns an interesting
strand of argument in the book that plays out in part II concerning justification
of belief in the existence of large cardinals. 

One possible path to justification here, that has its roots in ideas of G\"odel, 
is that one may be able to justify belief in large cardinals by 
a kind of intra-matheamtical indispensability argument. The idea is to show that there
is a significant class of important `low level' mathematical problems 
(say at the level of the integers) that are unsolvable except
by appeal to large cardinals, and then to defend belief in large cardinals 
on grounds they are needed to solve the relevant problems (or even just 
allow for simpler arguments than are otherwise available).\footnote{
    Of course Hellman himself, like Putnam in fact, 
    ultimately finds such arguments dubious on grounds 
    that abstract mathematical objects are not needed anywhere
    in light of the possibility of a modal-structural translation 
    of mathematics.
} 
In its initial G\"odelian form, this idea was developed using consistency statements 
for theories of interest as the `important mathematical problems' left open by set theory
without large cardinals but solvable in set theory with them. This initial form is 
open to natural concerns about whether or not consistency statements are `mathematical'
enough to constitute `important mathematical problems'; but, since G\"odel,
many further results have been found that do better on this score, including e.g. the 
Paris-Harrington theorem, various finite forms of Ramsey's theorem, and so on. At the 
cutting edge of these we have various combinatorial results about the integers and rationals
which Harvey Friedman has shown to be equivalent to farily strong large cardinal 
axioms, including the case highlighted by Hellman in which a statement about the integers 
in the field of Boolean relation theory is shown to (in a sense!) require the existence 
of an $n$-Mahlo cardinal for each $n$ to be proven.
The `G\"odel-Friedman', as Hellman 
calls it, aims to marshal such results as evidence in favor of the existence 
of the corresponding large cardinals.


Now I parenthetically remarked `in a sense' above, for the reason that
it is only in terms of \emph{consistency strength} that large cardinals are needed. 
In the case I just mentioned, for example, 
it is true that the proposition in Boolean Relation theory 
follows from the assumption of the existence of an $n$-Mahlo cardinal for each $n$, 
and that it will not follow from a weaker large cardinal axiom because the proposition 
itself implies the consistency of the claim, of each particular $n$, that an $n$-Mahlo 
exists (in fact it even implies the consistency of the \emph{schema} in $n$ asserting this).
Now this gives a \emph{sense} in which the relevant large cardinal hypothesis is necessary
in proving the proposition.  But 
of course this \emph{does not} mean the large cardinal hypothesis really is
a necessary condition for the proposition, since in fact it is 
enough simply to assume that the relevant hypothesis
\emph{does not have any false arithmetic consequences}, which is a much weaker 
assumption and itself a statement at the level of the integers. 
And it is completely obvious that this situation is general: whenever you have 
some statement at the level of the integers that seems to need a large cardinal 
axiom to prove, one will be able to get by with the much weaker assumption of 
conservativeness over the arithmetic fragment. Large cardinal hypotheses don't need
to be true to be useful, they just need to not mislead.
In his work, Solomon Feferman repeatedly appealed to such observations 
as a way of `defanging' the G\"odel-Friedman program.

The last two papers of part II, and especially the last one, contains the germ
of an idea for how to overcome this kind of problem by appeal to Bayesian 
confirmation theory. The central suggestion is that the famous `Bayes rule'

\begin{equation}
P(H|E) = \frac{P(E|H)\cdot P(H)}{P(E)}
\end{equation}

may somehow be invoked to give a `boost' to the subjective probability 
of large cardinal hypotheses, even given the foregoing observations.

Hellman's idea is to appeal to results like those of Friedman 
to `increase the posterior rational credence' of the hypothesis that large 
cardinals exist, \emph{via}  Bayes rule. According to Hellman, 
in the role of evidence (the $E$ in Bayes' rule) we may take syntactic 
claims like (say) the schema (in $n$) asserting the existence of an $n$ Mahlo cardinal
only has true arithmetic consequences; this, says Hellman, is a proposition to which 
we have `epistemic access', `arising in a variety of contexts' in such a way that
plausibly we can hope to `confirm... it through its own fruitfulness in other applications'.
(This is in contrast e.g. to Friedman's proposition from Boolean relation theory,
which Hellman does not think has these properties.) Hellman adds that besides this
consistency claim, in the case at hand, we may throw in additionally `consistency 
properties of other cardinals whose existence is derivable from the Mahlos', 
including inaccessibles, hyper-inaccessibles and so on -- 
to further enrich the evidence. In the role of the hypothesis $(H)$ of course
we will just have the large cardinal hypothesis in question, namely 
the existence of an $n$-Mahlo cardinal for each $n$. 

Hellman doesn't run through how the application of Bayes rule in 
such cases is supposed to go in detail, but I think it is illuminating to do so 
-- and reveals why application of the rule in mathematical cases in particular 
is perhaps more problematic than traditional empirical applications. 
In this case, it is clear that $P(E|H)$ is 1, so this drops out. 
We are left with the claim that one's degree of belief in the claim that 
the $n$-Mahlo cardinals exist, given they are arithmetically conservative, should be at least 
the result of dividing one's degree of belief that they exist by 
one's degree of belief that the corrsponding scheme is arithmetically conservative. (Technically
one could add to the $E$ side the other consistency properties mentioned by Hellman,
but these would be redundant for the obvious reason that they are all implied by 
the consistency property already mentioned, and so conjoining them
will not alter the probability of $E$.)

At this point we run into a bit of a problem. If one's degree of belief 
in the arithmetical conservativeness of the relevant scheme is high -- which for most people 
of course, it is -- then updating on this will make little difference to the 
posterior on their existence. Few people then will have their credences in 
Mahlos altered by this line of reasoning. Certainly, a skeptic like Feferman already
willing to concede consistency, but doubtful only of existence, will be little moved.

Now to \emph{some} extent this might be chalked up simply to the problem of 
`old evidence', familiar from more general discussions of Bayseianism. This is the 
problem that for Bayesian update to have \emph{any} effect, there must be a 
change in view on the $E$; after you have attained your evidence,
the formula will no longer work to tell you how to update.
Consider for example $E =$ `the die comes up 6 6 times', 
$H =$ `the die is loaded', against some background information $B$. Suppose 
the die has been rolled six times, and came up each time 6; and you are basically 
certain of this. Then applying 
Bayes rule will yield the wrong results; really you want $P(E) = 1/6^6$, 
not close to $1$, to get an appropriate update. 
In this case, as with such mundane cases more generally, 
it is obvious what to do: just imagine yourself in a situation before you were 
certain in $E$, before you got the relevant evidence, and reason from there to work
out how you should update now. Perhaps then we ought to do the same in the case 
of these consistency statments?

It is a little unclear how this would go: after all, it 
is not as though we can all remember a time before we thought Mahlo cardinals 
were arithmetically conservative, along with an `ah ha!' moment where the evidence was revealed. 
This points to a general issue in using 
Bayesian techniques for \emph{a priori} questions; it is not clear in 
these circumstances that we have a relevant notion of `evidence'.
If Hellman's proposal is to work, something more subtle will have to be invoked.

One natural and interesting idea here, suggested (in my opinion all too) 
briefly by Hellman, is instead of thinking in terms of \emph{an indvidual's} priors 
and posteriors we think in more general terms about
something more nebulous, like `the community of mathematicians', considered as a 
group over time. Here we can think of this group as accruing evidence for the 
consistency \& conservativeness of large cardinals gradually and over time through experience
of giving proofs in the relevant systems. On this way of thinking of things, if there 
were a case where the community as a whole were first rather skeptical of the 
consistency/conservativeness of some large cardinal statement, and then over time gradully came to 
be confident in the consistency, this could be taken to warrant Bayesian update 
among members of the community giving a boost in favor of the \emph{existence} of 
large cardinals, by a pattern of reasoning not dissimilar to that sketched in the case of the 
die.

The extent to which this sort of thinking \emph{should in fact} lead to a boost 
in confidence in large cardinals -- even if only under rational reconstruction --
is an interesting question. By dint of the way it is set up, it will be sensitive to 
the question of who gets to count as part of the `community'. (If we start with 
Aristotle, we may get one set of results; 
if we start with Cantor, things will look quite different.) It will also 
be sensitive to the large cardinal axiom. Plausibly, the arithmetical conservativeness of 
Mahlos has been a shock to few. But there are certainly 
interesting cases here: for example, it is currently considered (so far as I 
know) a \emph{very} open question whether the choiceless cardinals are consistent/conservative; 
so if over time, inductive familiarity led to a belief in consistency, this Bayesian 
machinery would predict we should be more confident in the \emph{truth} of the 
choicless cardinal axioms as well. 

This is an interesting potential result and the general framework of 
Bayesianism is in my opinion interesting ground in the epistemology of set theory,
ground that has been underexplored. Hellman's paper is a welcome remedy to this,
but still (naturally) leaves many questions unanswered and much work to do. Despite 
what seem to me to be unresolved issues in how to really carry through the application
of Bayesianism in this area, Hellman deserves huge credit for raising the idea 
and charting out the basic terrain.

%
\end{document}