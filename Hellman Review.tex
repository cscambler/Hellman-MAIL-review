\documentclass{asl}

\title{Review: Geoffrey Hellman, \emph{Mathematics and its Logics}}

\author{Chris Scambler}
\revauthor{C. Scambler}
\address{All Souls College\\
University of Oxford}
\email{cscambler@gmail.com}

\usepackage{tikz}
\usepackage{bussproofs}
% These will be typeset in italics
\newtheorem{Theorem}{Theorem}[section]
\newtheorem{Proposition}[Theorem]{Proposition}
\newtheorem{Lemma}[Theorem]{Lemma}
\newtheorem{Corollary}[Theorem]{Corollary}
\usepackage{times}
% These will be typeset in Roman
\theoremstyle{definition}
\newtheorem{Definition}[Theorem]{Definition}
\newtheorem{Fact}[Theorem]{Fact}
\newtheorem{Conjecture}[Theorem]{Conjecture}
\newtheorem{Remark}[Theorem]{Remark}


\begin{document} 
\maketitle
%\begin{abstract} \end{abstract}

\section{Introduction}
I have now made a change
%paragraph 1: introduction and compliments.
\emph{Mathematics and its logics} is a collection of Geoffrey Hellman's papers. 
The volume covers a huge variety of topics:
there are essays on the foundational role of category theory, 
on the nature and value of restrictive programs in foundations of mathematics such as nominalism and predicativism, 
on modal structuralism, on the Quine-Carnap debate, 
on intuitionism and smooth infinitesimal analysis, 
and much more besides. 
Hellman's important contributions on all of these topics are reflected in the collected papers, 
and in addition they tend to contain excellent introductions to the relevant debates along the way.
As a result, 
the book would make a valuable addition to the library of any graduate student or professional with a specialization in philosophy of mathematics. 

%paragraph 2: explain structure of review
The book is split into three parts reflecting different aspects of Hellman's work. 
Each part consists of a series of papers with a common theme.
Most of the papers are very well known, have been highly influential, 
and have already been subjects of much discussion in the literature.
However, in addition, 
each part also offers a newly written paper.
One of these is a critical piece, responding to recent work by Penelope Maddy. 
The other two push at the frontiers of the existing work showcased in parts one and two.
Since the older works are already so well known and have been so extensively discussed, 
after summarizing the papers offered in each part, 
in this review I will go on to discuss the contrubtions made by the two new non-critical papers.

\section{Overview}
%paragraph 3: overview part I
Part I consists in papers related to Hellman's modal structuralism and surrounding issues.
The first paper of part I, \emph{Structuralism without Structures}, 
is one of the founding works in the theory of modal structuralism;
in it, mathematical foundations are developed in detail for many areas of mathematics
in modal structuralist terms, and the philosophical credentials of modal structuralism are assessed.
(The paper also contains interesting points of comparison between modal structuralism and the predicativst viewpoint, 
which is itself the topic of scrutiny in part II.) 
The second paper, 
\emph{What Is Categorical Structuralism?}, 
presents a comprehensive discussion of the role of category theory in foundations of mathematics, 
arguing 
(against Colin McClarty and Steve Awodey) 
that ultimately category theory is in an important sense lacking in this regard. 
Papers three and four, 
\emph{On the Significance of the Burali Forti Paradox} 
and \emph{Extending the Iterative Concept of Set}, 
discuss the set-theoretic paradoxes from the potentialist point of view, 
making the case that the `height potentialist' perspective implied by modal structuralism allows for a more satisfying solution 
to the set-theoretic paradoxes,
 as well as clearer justifications for certain standard axioms of set theory, 
than are afforded by the more common `actualist' view. 
Papers five and six are related to nominalism, 
and (among other things) are concerned to defend the particular form of nominalism implied by his modal structuralism.

Part II offers a series of four papers 
exploring the nature and value of predicative foundations for mathematics. 
The first paper, \emph{Predicative Foundations of Arithmetic}, 
is joint with Solomon Feferman. 
It develops the first-order theory of arithmetic and proves its categoricity 
using an axiom system they call $\mathsf{EFSC}$, 
the `elementary theory of finite sets and classes',
which is argued also to be predicatively acceptable.
The framework developed presents a beatuifully economic account 
of the natural number structure from a remarkably austere collection of principles. 
In addition, the presented results may also be of relevance to contemporary discussions in philosophy.
The recent discussion of `internal categoricity' results and their philosophical significance, 
for instance, 
tends to take it as read that \emph{impredicative} comprehension is needed
to secure categoricity in this way, 
whereas the work done by Feferman and Hellman
seems to suggest that one can do with less.
A little bit -- 
in the form of $\mathsf{EFSC}$ -- 
goes further than one might expect. 

The second paper of part II, 
\emph{Challenges to Predicative Foundations of Arithmetic},
is also joint with Feferman.
It considers various philosophical objections raised to predicativism in the light 
of the mathematical work done in the first paper, 
and also presents some further technical developments contributed by Peter Aczel. 
The third paper, \emph{Predicativsm as a Philosophical Position},
is sole-authored. 
Perhaps unsurprisingly, 
it presents a more critical examination of predicativism as a philosophical position,
arguing that ultimately the central contribution of predicative foundations
is to emphasize the 
\emph{dispensability} of the uncountable from the point of view of modern science,
rather than providing an independently coherent and plausible foundation for mathematics.
The fourth paper, \emph{On the G\"odel-Friedman program}, 
is newly written for the volume.
It continues the analysis of the previous article in a more general context,
in particular that of Harvey Friedman's program for justifying large cardinal axioms 
by appeal to their utility in solving undecidable problems at the level of the naturals. 
Beyond the insightful discussion of Friedman's program, 
the article moves into novel and interesting territory regarding
the epistomology of large cardinals,
and in particular sets out the outlines of a Bayesian approach to justification here.

Part III is a little less homogeneous than the previous two parts. 
It presents five papers in the domain of philosophical logic. The first, 
\emph{Logical Truth by Linguistic Convention}, 
offers a defense of the Carnapian thesis 
that first-order logical knowledge is in a significant sense `analytic'. 
The second, \emph{Never say ``Never''!}, 
argues that intuitionism is an important sense not self-contained, 
since it turns out to rely on a `non-intuitiontistic' understanding of infintary quantification.
The third paper, 
\emph{Constructive Mathematics and Quantum Mechanics: Unbounded Operators and the Spectral Theorem} 
continues the assault on intuitionism by arguing that constructive mathematics 
is incapable of proving results needed in the foundations of quantum mechanics, 
especially the so-called `spectral theorem'. 
Paper four, \emph{If ``If-Then'' then what?}, discusses Maddy's recent attempts to revive (and extend) 
the classical position of `If-Then-ism' in foundations of mathematics.
Finally, the fifth paper, 
\emph{Mathematical Pluralism: The Case of Smooth Infinitesimal Analysis}, 
discusses the issue of how to understand the logical connectives in SIA, 
and argues that ultimately they must be understood as only as implicitly defined 
relative to a presupposed classical meta-theory, 
thus amplifying the theme in this part that intutionism and intuitionistic theories are, 
taken independently of a classical meta-translation, 
ultimately inadequate.
\section{Scraps}
I want to raise some issues surrounding the fourth paper of this part, which is newly written for the volume. In seminal work CITE, George Boolos showed how some standard axioms of set theory could be recovered from a theory of sets according to which they are ``formed in stages''. The paper was highly influential in the development of the iterative concept of set and (I believe) its uptake among philosophers. However, Boolos' paper argues that the axiom of replacement is special in that it is not straightforwardly recoverable from the stage theory. Similarly, the axiom of infinity cannot be recoverable from the pure stage theory, and so the existence of an infinite stage must simply be demanded as an extra axiom. 

The fourth paper argues that the `height potentialist' framework, something implied by Hellman's modal structuralism, allows an explicit argument for both the axiom of infinity and the axiom of replacement from the point of view of the iterative concept of set. I want now to briefly discuss this argument, and raise some questions about it. For the sake of simplicity, I will focus my discussion on the axiom of infinity, though everything I say also applies to replacement.

Recall that a \emph{height actualist} in set theory is someone who thinks there are some sets (plural) such that there can be no set of higher rank than all of them; in terms of the stage theory, this amounts to the claim that there are some stages in the process of set formation such that there can be no stage after all of them. Intuitively speaking, the sets/stages in question represent the sets/stages of The Cumulative Hierarchy, the structure containing all possible sets. Height actualism was surely held, implicitly at least, by Boolos and many after him. A \emph{height potentialist}, on the other hand, is someone who thinks that there can always be sets of higher rank, or that there can always be further stages in the process of set formation. For the height potentialist, there is no such thing as The Cumulative Hierarchy, since any cumulative hierarchy structure can be extended to include further stages of set formation.

Now, how does height potentialism help with the derivation of the axioms of infinity? First, we should note that the standard axiom of infinity -- that there \emph{exists} an infinite set, or least infinite ordinal, or whatever -- is \emph{not} Hellman's target; rather, he aims to show that one can establish the \emph{possible} existence of an infinite set in a potentialist stage theory. Effectively, Hellman's claim is that there are intuitively plausible axioms one can adopt in the potentialist setting that allow one to derive the possible existence of an infinite set. Of course, for this to work, the offered axioms have to be more intrinsically plausible than the simple assertion that infinite sets / stages in the process of set formation are possible.

Hellman considers one axiom of this kind that he ultimately rejects on grounds of its being `too close' to the sought conclusion in this regard. This is the assertion:
\begin{description}
\item[A] It is possible that there are some stages $ss$ such that the empty set is formed at one of them, and every stage in the $ss$ is succeeded by another stage in the $ss$. 
\end{description}
In the context of the rest of the stage theory, this immediately implies there is a stage with infinitely many stages before it, and hence an infinite set. But, according to Hellman, this axiom is 
\begin{quote} ``Too close for comfort to the conclusion sought. Once you have an infinity of compossible things... not surprisingly you generate a set of those things at the next stage.''\end{quote}
What then is Hellman's alternative? To introduce it, we need the following definition. Say that a property $P(ss)$ that applies to stages $ss$ is \emph{Indefinitely Extendible} (IE) iff each of
\begin{description} \item[B] $P(ss)$ is possible,
\item[C] given any possible stages $ss$ with $P(ss)$, it is possible to find a $t$ after all $ss$ with $P(ss + t)$
\end{description}
are true. In the latter of course $ss + t$ is just the $ss$ together with $t$ (which will be supplied by our plural comprehension axioms). The rough idea is that $P$ is an IE property when it applies to \emph{unboundedly many}  possible series of stages: so no matter how far you go on in the process of set formation, you will always eventually meet find new series of stages that satisfy $P$.

Now, let $N$ be the predicate
\begin{quote} Every stage in the $ss$ is one at which a natural number is formed
\end{quote} 
Then the axiom Hellman proposes is simply that `$N$ is not IE'. Let us see what this amounts to. Substituting $N$ for $P$ in $IE$ and negating we get the negation of one of {\bf B} or {\bf C} with $N = P$. The resulting negation of {\bf B} is demonstrably false given the rest of Hellman's stage theory, so we are left with (up to trivial grammatical reshuffling)
\begin{description}\item[D] possibly, there are stages $ss$ where every one of $ss$ witnesses the formation of a new natural number, and such that no later stage witnesses the formation of a natural number.
\end{description}
Using the fact that the rest of Hellman's theory implies it is possible for there to be a stage $s_n$ at which each particular natural number $n$ is formed, one can easily go on to show that there must possibly be a stage at which the set of all natural numbers is formed (and much more besides). But this is the potentialist version of the axiom of infinity we set out to secure, that it is possible there is an infinite set.

I have three issues with this argument. The first is that it is not obvious to me that there is much daylight between Hellman's proposed axiom and axiom {\bf A}. The second is that the machinery of stage theory and the Boolosian iterative conception seems superfluous, and indeed unnecessarily complicating. The third is that, given the difference in target in Hellman's derivation from the standard axiom of infinity, it is in a certain sense trivial that his conclusion is correct, \emph{viz.} that there are arguments available for infinity to the height potentialist not available to the actualist.

On the issue of daylight between {\bf A} and Hellman's proposal: Let's review the dialectic. Hellman's stage theory is committed to \begin{description} \item[E] For each natural number $n$, it is possible to find a stage $s_n$ at which the finite Von Neumann ordinal $n$ is formed.\end{description} 
The question is whether in addition it is possible to form all such stages `at once'. {\bf A} says `yes' quite straightforwardly: it asserts that it is possible to run through all the finite stages of the set construction process. In the context of {\bf E} on the other hand, {\bf D} is a way of saying `yes' bentbackwardly, by saying that eventually the stages that witness formation of natural numbers \emph{stop coming}, even though the stages continue. But (given that by {\bf E} such stages do come for each $n$) this really seems a trivial restatement of the claim that eventually you get all of them.  In fact they seem roughly akin to the pair `someone is wearing a hat' and `not everyone is not wearing a hat': you can either say eventually you get all of the stages at which natural numbers are formed, or that eventually you won't get any more stages at which natural numbers are formed, but in either case you are saying essentially the same thing.

As to the second point it is a little unclear why talk of stages should be relevant to such arguments for the justification of infinity, and in fact there are simpler though intimately related proposals known in the literature. For example, a simpler definition of an IE predicate says that the predicate $P$ is IE if necessarily, given any plurality $pp$ of $P$-ers, it is possible to find a $p$ not among the $pp$ with $P(p)$. Then the axiom that says the predicate `is a natural number' or `is a hereditarily finite set' is not indefinitely extendible seems to have all the plausibility of Hellman's proposal but without need to refer to stages. In fact, this (more or less) is the statement of the axiom of infinity in Linnebo's potentialist system.

With regard the third point, we already noted that Hellman's target is not what is usually associated with the axiom of infinity: standardly, this would be understood as something implying the \emph{existence} of an infinite set. None, however, of Hellman's potentialist stage theories implies anything about simple existence; they only require it to be \emph{possible} for there to be an infinite set (or stage). When he speaks of `deriving the axiom of infinity', it is this weaker sense -- deriving the \emph{possibility} of an infinite set -- that he has in mind. As a result there is clear sense in which the potentialist -- regardless of whether or not they talk directly in terms of stages -- will have route to justifying what they call the axiom of infinity not open to the height actualist: they only need to argue for the \emph{possibility}, in whatever sense is at issue, of the actualist's axiom. If logical possibility is at issue, for example, they only need argue for something along the lines of \emph{consistency} of the actualist's axiom to secure its \emph{truth}. And clearly, in general, the potentialist's claim be weaker and so easier to justify. But this is just a trivial consequence of the fact that they mean different things by the axiom of infinity.

The second part consists in papers, some of which are joint with Solomon Feferman, that explore the nature and value of predicative foundations for mathematics. The first paper, which is joint with Feferman, presents technical underpinnings for a predicativist theory based in a modest and elegant theory of finite sets. The second, also joint with Feferman, considers some philosophical challenges raised to the account of the first paper, and also presents some further technical developments contributed by Peter Aczel. The third paper begins a philosophical analysis of the predicativist stance, and the fourth, newly written for the volume, continues that analysis, and in addition fleshes out an idea aired briefly in the third paper, namely the application of Bayesian epistemology in philosophy of mathematics as a means for justifying belief in the existence of large cardinals. The part as a whole is remarkable in showcasing the collaborative and interdisciplinary research involving mathematicians and philosophers that makes philosophy of mathematics such an exciting field.

I want here to briefly discuss the final topic, the use of Bayseian confirmation theory to support large cardinal axioms.

In the background here is an interesting dialectic regarding justification for existence of large cardinals \emph{via} their use in solving `low level', generally arithmetic, problems. The oldest idea of this kind is G\"odel's idea about incompletenes. For example, ZFC does not prove its own consistency, a statement at the level of the natural numbers, but ZFC + there exists an inaccessible does. Thus we are compelled to `move up' the cumulative hierarchy via large cardinal axioms to solve independent problems even at the arithmetic level. More sophisticated examples have since been discovered, of course, including the Paris Harrington theorem, a finite form of Kruskal's theorem, and the litany of results of this kind offered by Harvey Friedman.

All these results, however, suffer from the same problem when viewed as arguments for the indispensability of large cardinals: namely, the \emph{existence} of the large cardinals in question is in general sufficient but unnecessary; for their mere \emph{conservativeness over the arithmetic fragment of the language}, the schema $Prov_T(A) \rightarrow A$ for arithmetic $A$, will on the one hand allow for the desired results, but on the other itself remains a statement about provability at the level of the natural numbers, not something committed to large cardinals.

A suggestion in the 3rd and 4th papers of the part is that Bayesian confirmation theory will be relevant. This is an interesting suggestion and one that is well worth further exploration. At the same time, in my view, much further exploration is needed, since there remain significant questions about the utility of the Bayesian approach here that are not answered in the paper.

The central suggestion is that confirmation theory and especially the famous `Bayes rule'
\begin{equation}
P(H|E) = \frac{P(E|H)\cdot P(H)}{P(E)}
\end{equation}
may somehow be invoked to give a `boost' to the subjective probability of large cardinal hypotheses, even if of course that boost will fall short of the security provided by a deductive proof. This idea has roots, according to Hellman, in a famous remark of G\"odel's, in which G\"odel proposes that large cardinal axioms and other axioms extending ZFC may be justified in terms of their `success' and `verifiable consequences'.

Hellman's idea is to appeal to results like those of Friedman to `increase the posterior rational credence' of the hypothesis that large cardinals exist, \emph{via}  Bayes rule. According to Hellman, in the role of evidence (the $E$ in Bayes' rule) we may take syntactic claims like that (say) Mahlo cardinals are 1-consistent, and in the role of the hypothesis  $(H)$ we can plug in the existence of Mahlo cardinals themselves. [[Actually it is a bit unclear what he is getting at here. How is the Friedman stuff involved?]]

Hellman never actually runs through how the application of Bayes rule in such cases is supposed to go, but I think it is illuminating to do so -- and reveals why application of the rule in mathematical cases in particular is perhaps more problematic than traditional empirical applications. In this case, it is clear that $P(E|H)$ is 1, so this drops out. We are left with the claim that one's degree of belief in the claim that Mahlo cardinals exist given they are 1-consistent should be at least the result of dividing one's degree of belief that they exist by one's degree of belief that they are 1-consistent.\footnote{There is also the issue of more varied evidence, but...}

At this point we run into a bit of a problem. If one's degree of belief in the 1-consistency of Mahlos is high -- which for most people of course, it is -- then updating on this will make little difference to the posterior on their existence. Few people then will have their credences in Mahlos altered.

Now to \emph{some} extent this might be chalked up simply to the problem of `old evidence', familiar from discussions of Bayseianism. This is the problem that for Bayesian update to have \emph{any} effect, there must be a change in view on the $E$. Consider for example $E =$ `the die comes up 6', $H =$ `the die is loaded', against some background information $B$. Suppose the die has been rolled; you are basically certain its a six. Then applying Bayes rule will yield the wrong results; really you want $P(E) = 1/6$, not close to $1$. In this case, and with other similar straightforward cases, it is obvious what to say: just imagine yourself in a situation before you were certain in $E$, before you got the relevant evidence, and reason from there. But in the mathematical case, the same move looks a little less plausible. It is not as though we can all remember a time before we thought Mahlo cardinals were consistent; most people probably have taken it as read since they were introduced to them. It is even fairly unclear what might motivate an upward shift in one's credence in the consistency of Mahlos (although sharply downward is easy enough to conceive).

One natural idea suggested briefly by Hellman is that evidence comes from familiarity with the systems; ones confidence in the consistency of a system is gradually increased as one uses it, becomes familiar with it, all the while not finding any contradictions. Perhaps this is so, but in the case he discusses (Mahlo cardinals) it seems highly implausible that these kinds of enquires have significantly raised peoples confidence. Similarly inaccessibles.

Perhaps, however, there are some such cases. Measurables were widely held to be inconsistent but gradual familiarity seems to have persuaded people otherwise. Similarly with the hierarchy of choiceless cardinals. Initially suspected to be consistent, but pursuit revealed rich structure theory. Increasingly people think they are consistent. But this consistency \emph{is} surprising. So perhaps people are more willing to take the case for these cardinals -- and hence the case for the ultimate falsity of AC -- more seriously. In this case, Hellman's Bayseian model does seem to apply in a potentially interesting way.



The third part consists in papers on philosophical logic, generally united by the fact that the issues discussed have some bearing on foundations for mathematics. Explain.

Really good book. To quote Hellman on Burgess and Rosen, philosophy at its best!




\bibliography{masterbib}
\bibliographystyle{asl}



\end{document}